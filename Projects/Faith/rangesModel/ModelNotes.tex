%document describing teh final version of ny dosse response model and what I think about its fit. Writen by Faith, started Augst 21st 2020.
\documentclass[11pt,letter]{article}
\usepackage[top=1.00in, bottom=1.0in, left=1.1in, right=1.1in]{geometry}
\renewcommand{\baselinestretch}{1.1}
\usepackage{graphicx}
\usepackage{natbib}
\usepackage{amsmath}
\usepackage{textcomp}%amoung other things, it allows degrees C to be added
\usepackage{float}
\usepackage{hyperref}
\usepackage[utf8]{inputenc} % allow funny letters in citaions 
\usepackage[nottoc]{tocbibind} %should add Refences to the table of contents?
\usepackage{amsmath} % making nice equations 

\def\labelitemi{--}
\parindent=0pt

\title{Cat's Ranges Model \\ A 3 level Hierarchical model with ncp}
\date{\today}
\author{Faith Jones}

\graphicspath{ {./DoseResponse_WriteupImages/} }% tell latex where to find photos 


\begin{document}
%\renewcommand{\refname}{\CHead{}}%not sure what this was supposed to do 
\renewcommand{\bibname}{References}%names reference list 


\maketitle{}
\tableofcontents

\section{Data and question}
This model takes what I assume is OSPREE forcing and photoperiod data for lots of studies and species. There is an additional level of variation in this data, which is the population the species comes from. I think it is possible for different species to come from the same population, but Cat you should correct me if I'm wrong. The question we focus on is whether there is greater variation within a species (in terms of variation between populations) than between species. 

\section{Model}

This model differs a bit from Cat's original one. Maybe we will get time to discuss how they differ at the end of the meeting. The original code is in \textit{stan\\/nointer\_3levelwpop\_force\&photo\_ncp.stan}, and the modified code I refer to in this documents is in \textit{stan\\/nointer\_3levelwpop\_force\&photo\_ncp\_FaithExample.stan}


\begin{equation}
\tilde{y}_{i}\sim normal(\mu_{i},\sigma_{y})
\end{equation}


\begin{equation}
\mu_{i} = \alpha + \alpha_{study} + \alpha_{sppop} + force_{i} * \beta _{force,sppop} + photo_{i} * \beta _{photo,sppop} 
\end{equation}

Study intercept 
\begin{equation}
\alpha_{study} =  \sigma_{\alpha, study} * \alpha raw_{study}
\end{equation}

Species intercept
\begin{equation}
\alpha_{species} = \sigma_{\alpha, species} + \alpha raw_{species}
\end{equation}

Subpopulation intercept 
\begin{equation}
\alpha_{sppop} = \alpha_{species}  + \sigma_{\alpha ,sppop} * \alpha raw_{sppop}
\end{equation}

Species forcing slope 
\begin{equation}
\beta _{force,species} = \beta_{force} + \sigma_{beta force, species} * \beta raw_{force}
\end{equation}

Subpopulation forcing slope 
\begin{equation}
\beta _{force,sppop} = \beta _{force,species} + \sigma_{\beta force,sppop} * \beta raw_{force,sppop}
\end{equation}

Species photoperiod slope 
\begin{equation}
\beta_{photo,species}  = \beta_{photo} + \sigma_{beta photo, species} * \beta raw_{photo}
\end{equation}

Subpopulation photoperiod slope 
\begin{equation}
\beta _{photo,sppop} = \beta_{photo, species} + \sigma_{\beta photo,sppop} * \beta raw_{photo, sppop}
\end{equation}



\section{Priors}

Grand intercept 
\begin{equation}
\sigma_{\alpha, study} \sim normal(0,20)
\end{equation}

Study intercept 
\begin{equation}
\sigma_{\alpha, study} \sim normal(0,20)
\end{equation}

\begin{equation}
\alpha raw _{study} \sim normal(0,1)
\end{equation}


Species intercept
\begin{equation}
\alpha raw _{species}\sim normal(0,1)
\end{equation}

\begin{equation}
\sigma_{\alpha, species} \sim normal(0,20)
\end{equation}


Subpopulation intercept 
\begin{equation}
\sigma_{\alpha, sppop} \sim normal(0,20)
\end{equation}

\begin{equation}
\alpha raw _{sppop}\sim normal(0,1)
\end{equation}


Grand forcing slope
\begin{equation}
\beta_{force}  \sim normal(0,20)
\end{equation}

Species forcing slope 
\begin{equation}
\beta raw_{force}\sim normal(0,1)
\end{equation}

\begin{equation}
\sigma_{\beta force,species}
\end{equation}

Subpopulation forcing slope 
\begin{equation}
 \sigma_{\beta force,sppop} \sim normal(0,20)
\end{equation}

\begin{equation}
\sigma_{\beta force,sppop}
\end{equation}

Grand photoperiod slope 
\begin{equation}
\beta_{photo}  \sim normal(0,20)
\end{equation}

Species photoperiod slope 

\begin{equation}
\sigma_{\beta photo,species}
\end{equation}

\begin{equation}
\beta raw_{photo}\sim normal(0,1)
\end{equation}

Subpopulation photoperiod slope 
\begin{equation}
\sigma_{\beta photo,sppop}
\end{equation}

\begin{equation}
\beta raw_{photo, sppop}\sim normal(0,1)
\end{equation}

General variance 

\begin{equation}
\sigma{_y}\sim normal(0,10)
\end{equation}




\section{Model explained in words}



\begin{equation}
\tilde{y}_{i}\sim normal(\mu_{i},\sigma_{y})
\end{equation}

Every $\tilde{y}_{i}$ value is centered around a mean predicted value $\mu_{i}$ with a normal distribution of width $\sigma_{y}$

\begin{equation}
\mu_{i} = \alpha + \alpha_{study} + \alpha_{sppop} + force_{i} * \beta _{force,sppop} + photo_{i} * \beta _{photo,sppop} 
\end{equation}

Each mean predicted value $\mu_{i}$ has intercept which is a combination of a grand intercept ($\alpha$), an intercept based on the study of value $i$ ($\alpha_{study}$), and intercept variation due to the species and population combination ($\alpha_{sppop}$). There are also two slopes that relate forcing ($\beta _{force,sppop} $) and photoperiod ($\beta _{photo,sppop} $) of value $i$ to it's predicted $\mu_{i}$ value. These slope values depend on the species and subpopulation of value $i$, meaning a different species and subpopulation combination may react faster or slower to forcing and photoperiod. 

\begin{equation}
\alpha_{study} = \sigma_{\alpha, study} * \alpha raw_{study}
\end{equation}

Each study deviates somewhat from the grand mean intercept $\alpha$; we call this value $\alpha_{study}$. Sometimes you might see this written as  $\alpha_{study} \sim normal(\alpha, \sigma_{apha,study})$. 

\begin{equation}
\alpha_{sppop} = \alpha_{species}  + \sigma_{\alpha force,sppop} * \alpha raw_{force,sppop}
\end{equation}

Each population deviates somewhat from it's species mean value ($\alpha_{species}$). The amount they deviate a normal distribution with a width of $\sigma_{\alpha force,sppop}$. 

\begin{equation}
\alpha_{species} = \sigma_{\alpha, species} * \alpha raw_{species}
\end{equation}

How much each species deviates from the grand mean $\alpha$ value is called $\alpha_{species} $, and is drawn from a normal distribution of width $\sigma_{\alpha, species} $. 

\begin{equation}
\beta _{force,sppop} = \beta _{force, species} + \sigma_{\beta force,sppop} * \beta raw_{force,sppop}
\end{equation}

Each population's rate of change due to an amount of forcing, that is the slope of forcing, is somewhat different. We describe this as the population values being centered around it's species mean slope $ \beta _{force, species}$ and has a distribution of width $\sigma_{\beta force,sppop} $. 

\begin{equation}
\beta _{force,species} = \beta_{force} + \sigma_{beta force, species} * \beta raw_{force}
\end{equation}

Each species's slope value for teh effect of forcing is drawn from a normal distribution centred around the grand beta slope $\beta_{force}$ and a width of $\sigma_{beta force, species}$.

\begin{equation}
\beta _{photo,sppop} = \beta_{photo,species} + \sigma_{\beta photo,sppop} * \beta raw_{photo}
\end{equation}

Each population's rate of change due to an amount of photoperiod, that is the slope of photoperiod, is somewhat different. We describe this as the population values being centered around it's species mean slope $ \beta _{force, species}$ and has a distribution of width $\sigma_{\beta force,sppop} $. 


\begin{equation}
\beta_{photo,species}  = \mu_{\beta photo, species} + \sigma_{beta photo, species} * \beta raw_{photo, sppop}
\end{equation}

Each species's slope value for the effect of photoperiod is drawn from a normal distribution centred around the grand beta slope $\beta_{force}$ and a width of $\sigma_{beta force, species}$.


\bibliographystyle{/home/faith/Documents/Bibtex/styles/besjournals/besjournals}



\bibliography{/home/faith/Documents/Bibtex/mendeleyStuff/library}

\end{document}

